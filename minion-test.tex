% !TEX TS-program = xetex
% !TEX encoding = UTF-8

\input minion-pro.tex

%% Źródło: http://monika.univ.gda.pl/~literat/sarag/0009.htm

%% fc-list | egrep -i 'minion|myriad' |sort
%%
%% Minion Pro:style=Bold
%% Minion Pro:style=Bold Italic
%% Minion Pro:style=Italic
%% Minion Pro:style=Regular
%% Myriad Pro:style=Bold
%% Myriad Pro:style=Bold Italic
%% Myriad Pro:style=Italic
%% Myriad Pro:style=Regular

% #1, #2, #3 -- tekst
% #4, #5, #6 -- math 
% #7         -- interlinia
\fontsetup{14}{11}{8}{14}{11}{8}{18}
\uselanguage{polish}

DZIEŃ ÓSMY

Ponieważ mam zaszczyt opowiadać wam moje przygody, łatwo więc
pojmiecie, że nie umarłem od wypitej - jak myślałem -
trucizny. Odszedłem tylko od przytomności i nie wiem, jak długo
zostawałem w tym stanie. Pamiętam jednak, że znowu obudziłem się pod
szubienicą Los Hermanos, ale tym razem z pewnego rodzaju zadowoleniem,
gdyż przynajmniej byłem pewny, żem jeszcze nie umarł. Nadto nie
znalazłem się już między dwoma wisielcami; leżałem po lewej ich
stronie, po prawej zaś spostrzegłem jakiegoś człowieka, którego także
wziąłem za wisielca, gdyż zdawał się bez życia i miał stryczek na
szyi. Wkrótce jednak poznałem, że śpi tylko, i rozbudziłem
go. Nieznajomy, spojrzawszy na miejsce swego noclegu, zaczął śmiać się
i rzekł:

- Trzeba wyznać, że w praktykach kabalistycznych wydarzają się czasem
przykre nieporozumienia. Złe duchy tyle umieją przybierać na się
kształtów, że nie można wiedzieć, z kim się ma do czynienia. Wszelako
- dodał - skądże mi się wziął ten powróz na szyi, w miejscu plecionki
z włosów, którą wczoraj jeszcze miałem na sobie?

Później, spostrzegłszy mnie, rzekł:

- I senor tutaj?... senor jeszcze za młody jesteś, jak na
kabalistę. Ale widzę, że także masz powróz na szyi.

W istocie, przekonałem się, że ma słuszność. Przypomniałem sobie, że
Emina wczoraj zawiesiła mi na szyi plecionkę z włosów swoich i
Zibeldy, i sam nie wiedziałem, jak sobie tę przemianę tłumaczyć.

Kabalista spojrzał na mnie bystrym wzrokiem i rzekł:

- Nie, ty do nas nie należysz. Nazywasz się Alfons, twoja matka rodzi
się z Gomelezów, jesteś kapitanem w gwardii wallońskiej, masz wiele
odwagi, ale mało doświadczenia. Mniejsza o to, trzeba naprzód stąd się
wydostać, a potem zobaczymy, co z sobą poczniemy.

Brama zagrody była otwarta, wyszliśmy i znowu ujrzałem przed sobą
przeklętą dolinę Los Hermanos. Kabalista zapytał mnie, dokąd chcę się
udać. Odpowiedziałem mu, że mam zamiar udać się drogą w kierunku
Madrytu.

- Zgoda - rzekł - ja także idę w tę stronę, ale zacznijmy naprzód od
przyjęcia jakiego posiłku.

[...]


OPOWIADANIE PASZEKA

Mój ojcze, właśnie znajdowałeś się w kaplicy i śpiewałeś litanie, gdy
posłyszałem kołatanie do drzwi i beczenie zupełnie podobne do tego,
jakie wydaje nasza biała koza. Myślałem więc, że zapomniałem ją wydoić
i poczciwe zwierzę przyszło mi przypomnieć mój obowiązek. Tym bardziej
trwałem w tym przekonaniu, że właśnie przed kilkoma dniami wydarzył mi
się podobny wypadek. Wyszedłem z chaty i w samej rzeczy ujrzałem naszą
białą kozę. która obracała się tyłem do mnie, pokazując mi nadęte
wymiona. Chciałem ją przytrzymać. aby jej oddać żądaną przysługę, ale
wymknęła mi się z rąk i, co chwila zatrzymując się i znów uciekając
dalej, zaprowadziła mnie nad brzeg przepaści, która otwiera się obok
twojej pustelni.

Przybywszy tam biała koza nagle przerzuciła się W czarnego
kozła. Przeląkłem się na widok tej przemiany i chciałem uciekać ku
naszemu mieszkaniu, ale czarny kozioł przeciął mi drogę i wspiąwszy
się na tylnych nogach, spojrzał na mnie płomiennymi oczyma. Strach
ściął mi lodem krew w żyłach.

\end
