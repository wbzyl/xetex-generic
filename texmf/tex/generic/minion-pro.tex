%% Font setup for Minion
%%
%% fc-list | egrep -i latin
%%
%% Latin Modern Roman,LM Roman 10:style=10 Bold Italic,Bold Italic
%% Latin Modern Roman,LM Roman 10:style=10 Regular,Regular
%% Latin Modern Roman,LM Roman 10:style=10 Italic,Italic
%% Latin Modern Mono,LM Mono 10:style=10 Regular,Regular
%% Latin Modern Mono,LM Mono 10:style=10 Italic,Italic
%% Latin Modern Roman,LM Roman 10:style=10 Bold,Bold
%%
%% fc-list | egrep -i 'minion|myriad' |sort
%%
%% Minion Pro:style=Bold
%% Minion Pro:style=Bold Italic
%% Minion Pro:style=Italic
%% Minion Pro:style=Regular
%% Myriad Pro:style=Bold
%% Myriad Pro:style=Bold Italic
%% Myriad Pro:style=Italic
%% Myriad Pro:style=Regular

%% \font \lmroman  "Latin Modern Roman/I" at 14pt
%% \font \rmMinion "Minion Pro/Regular:mapping=mex-text,+onum"
%% \font \itMinion "Minion Pro/Italic"
%% \font \bfMinion "Minion Pro/Bold"

\catcode`\@=11

\newdimen\normalparindent
\normalparindent=20\p@

%\newfam\slfam 

\def\fontsetup#1#2#3#4#5#6#7{%
% Text fonts
    \font\xtextroman="Minion Pro" at #1\p@
    \font\xxtextroman="Minion Pro" at #2\p@
    \font\xxxtextroman="Minion Pro" at #3\p@
%
    \font\xtextitalic="Minion Pro/I" at #1\p@
    \font\xxtextitalic="Minion Pro/I" at #2\p@
    \font\xxxtextitalic="Minion Pro/I" at #3\p@
%
    \font\xtextbold="Minion Pro/B" at #1\p@
    \font\xxtextbold="Minion Pro/B" at #2\p@
    \font\xxxtextbold="Minion Pro/B" at #3\p@
%
    \font\xtextteletype="Latin Modern Mono" at #1\p@
    \font\xxtextteletype="Latin Modern Mono" at #2\p@
    \font\xxxtextteletype="Latin Modern Mono" at #3\p@
% Math fonts
    \font\xmathtext=cmr10 at #4\p@ % must be tfm based font
    \font\xxmathtext="Latin Modern Roman/R" at #5\p@
    \font\xxxmathtext="Latin Modern Roman/R" at #6\p@
%
    \font\xmathitalic=lmmi10 at #4\p@
    \font\xxmathitalic=lmmi10 at #5\p@
    \font\xxxmathitalic=lmmi10 at #6\p@
%
    \font\xmathsymbol=lmsy10 at #4\p@
    \font\xxmathsymbol=lmsy10 at #5\p@
    \font\xxxmathsymbol=lmsy10 at #6\p@
%
    \font\xmathext=lmex10 at #4\p@
    \font\xxmathext=lmex10 at #5\p@
    \font\xxxmathext=lmex10 at #6\p@
%
    \font\xtextsansroman="Myriad Pro/BI" at #1\p@
%
    \def\SMC{\xxtextroman}%
    \def\ss{\xtextsansroman}
%
% Setting up font families
% % family 0
    \textfont0=\xmathtext
    \scriptfont0=\xxmathtext
    \scriptscriptfont0=\xxxmathtext
% family 1
    \textfont1=\xmathitalic
    \scriptfont1=\xxmathitalic
    \scriptscriptfont1=\xxxmathitalic
% family 2
    \textfont2=\xmathsymbol
    \scriptfont2=\xxmathsymbol
    \scriptscriptfont2=\xxxmathsymbol
% family 3   don't used in sup/sub scripts??
    \textfont3=\xmathext
    \scriptfont3=\xxmathext
    \scriptscriptfont3=\xxxmathext
% family 4
    \textfont\itfam=\xtextitalic
    \scriptfont\itfam=\xxtextitalic
    \scriptscriptfont\itfam=\xxxtextitalic
% family 5
%     \newfam\slfam \def\sl{\fam\slfam\tensl} % \sl is family 5
    \def\sl{\fam\slfam\tensl} % \sl is family 5
% family 6 (plain codes)
    \textfont\bffam=\xtextbold
    \scriptfont\bffam=\xxtextbold
    \scriptscriptfont\bffam=\xxxtextbold
% family 7
    \textfont\ttfam=\xtextteletype
    \scriptfont\ttfam=\xxtextteletype
    \scriptscriptfont\ttfam=\xxxtextteletype
% Font switches
    \def\rm{\fam0\xtextroman}%
    \def\mit{\fam\@ne}%
    \def\oldstyle{\fam\@ne\xmathitalic}%
    \def\cal{\fam\tw@}
    \def\it{\fam\itfam\xtextitalic}%
    \def\bf{\fam\bffam\xtextbold}%
%     \def\bfit{\fam\bolditalicfam\textbolditalic}%
    \def\bfit{\textbolditalic}%
    \def\tt{\fam\ttfam\xtextteletype}%
% Baselineskip, parindent, etc
    \normalbaselineskip=#7\p@% approx 3pt leading
    \smallskipamount=0.25\normalbaselineskip plus 1.5\p@ minus 1\p@%
    \medskipamount=0.5\normalbaselineskip plus 3.375\p@% minus 2\p@%
    \bigskipamount=#7\p@ plus 3\p@ minus 2\p@%
% Math 
    \abovedisplayskip=0.5\normalbaselineskip plus 0.25\normalbaselineskip
    \abovedisplayshortskip=0pt plus 0.25\normalbaselineskip
    \belowdisplayskip=0.5\normalbaselineskip plus 0.25\normalbaselineskip
    \belowdisplayshortskip=0.5\normalbaselineskip plus 0.25\normalbaselineskip
% Rest
    \parindent=\normalparindent % should no change
    \parskip=0\p@ plus 0\p@%
    \normallineskip=1\p@%
    \normallineskiplimit=\z@%
    \splittopskip=9\p@% strutt height should not change too??
    \setbox\strutbox=\hbox{\vrule height9\p@ depth5\p@ width\z@}%
    \normalbaselines\rm}

\def\roman#{\bgroup\rm\let\next= }
\def\bold#{\bgroup\bf\let\next= }
\def\ttype#{\bgroup\tt\let\next= }
\def\italic#1%
    {\begingroup
     \def \next
         {\def \next
              {\setbox 0 = \hbox {\nexttoken}%
               \ifdim \ht 0< \fontdimen 5 \font
                  \it #1%
               \else
                  \it #1\/%
               \fi}% setbox
          \ifcat A\noexpand \nexttoken
             \next
          \else 
             \ifcat 0\noexpand \nexttoken
                \next
             \else
                \ifcat \space \noexpand \nexttoken
                \else
                %%% \message {?: \meaning \nexttoken}%
                \fi
                \it #1\/%
             \fi
          \fi
          \endgroup}% next
     \futurelet \nexttoken \next}

\def\bolditalic#1%
    {\begingroup
     \def \next
         {\def \next
              {\setbox 0 = \hbox {\nexttoken}%
               \ifdim \ht 0< \fontdimen 5 \font
                  \it #1%
               \else
                  \bfit #1\/%
               \fi}% setbox
          \ifcat A\noexpand \nexttoken
             \next
          \else 
             \ifcat 0\noexpand \nexttoken
                \next
             \else
                \ifcat \space \noexpand \nexttoken
                \else
                %%% \message {?: \meaning \nexttoken}%
                \fi
                \bfit #1\/%
             \fi
          \fi
          \endgroup}% next
     \futurelet \nexttoken \next}

\catcode`\@=12

\catcode`\|=13
\def|{\begingroup\setupverbatim\doverbatim|}%

\endinput

% #1, #2, #3 -- text sizes
% #4, #5, #6 
% #7         -- baseline
\fontsetup{12}{10}{8}{12}{10}{8}{15}

$x-y$

$x+y$

$x=y$

$x\ne y$
