% !TEX TS-program = xetex
% !TEX encoding = UTF-8

\input mimulcol.tex % GUSTLIB
\input minion-pro.tex
\input test-abc.tex

% #1, #2, #3 -- text
% #4, #5, #6 -- math sizes
% #7         -- baseline
@fontsetup{12}{10}{8}{12}{10}{8}{16}

@ZESTAW JP / 12A
@WHERE  Instytut Informatyki

@beginmulticols 2

@PRINTHEADER

@Z Każdy ze studentów Informatyki ma założone konto pocztowe na
   serwerze {\it Sigma}. 
   Będąc zalogowani na \italic{Sigmie} możemy wysłać maila za
   pomocą:
@a+  programu |mutt|
@b+  edytora Emacs
@c-  programu |mailaddr|

@Z Polecenie |whoami| wypisze na ekran:
@#
@a+ login zalogowanego użytkownika
@b- loginy wszystkich zalogowanych w~laboratorium użytkowników
@c- loginy wszystkich zalogowanych użytkowników

@Z Po wykonaniu kodu:
@verbatim@@
char s1[] = "ala";
*(s1+1) = 'g';
@@
Tablica |s1| będzie równa:
@a- |ala|
@B+ |aga|
@C- |gla|

@fillbottom
@endmulticols

@PRINTNOKEYS

@PRINTKEYS

@end
