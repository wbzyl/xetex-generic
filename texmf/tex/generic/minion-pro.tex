\catcode`\@=11

\newdimen\normalparindent
\normalparindent=20\p@

\def\fontsetup#1#2#3#4#5#6#7{%
% Text fonts
    \font\xtextroman="Minion Pro:+onum" at #1\p@
    \font\xxtextroman="Minion Pro:+onum" at #2\p@
    \font\xxxtextroman="Minion Pro:+onum" at #3\p@
%
    \font\xtextitalic="Minion Pro/I:+onum" at #1\p@
    \font\xxtextitalic="Minion Pro/I:+onum" at #2\p@
    \font\xxxtextitalic="Minion Pro/I:+onum" at #3\p@
%
    \font\xtextbold="Minion Pro/B:+onum" at #1\p@
    \font\xxtextbold="Minion Pro/B:+onum" at #2\p@
    \font\xxxtextbold="Minion Pro/B:+onum" at #3\p@
%
    \font\xtextteletype="Latin Modern Mono" at #1\p@
    \font\xxtextteletype="Latin Modern Mono" at #2\p@
    \font\xxxtextteletype="Latin Modern Mono" at #3\p@
% Math fonts
    \font\xmathtext=plr10 at #4\p@ % must be tfm based font
    \font\xxmathtext=plr10 at #5\p@
    \font\xxxmathtext=plr10 at #6\p@
%
    \font\xmathitalic=plmi10 at #4\p@
    \font\xxmathitalic=plmi10 at #5\p@
    \font\xxxmathitalic=plmi10 at #6\p@
%
    \font\xmathsymbol=plsy10 at #4\p@
    \font\xxmathsymbol=plsy10 at #5\p@
    \font\xxxmathsymbol=plsy10 at #6\p@
%
    \font\xmathext=plex10 at #4\p@
    \font\xxmathext=plex10 at #5\p@
    \font\xxxmathext=plex10 at #6\p@
%
    \font\xtextsmc="Myriad Pro:+smc" at #1\p@
    \font\xtextsansroman="Myriad Pro" at #1\p@
    \font\xtextsansitalic="Myriad Pro/I" at #1\p@
%
    \def\smc{\xxtextsmc}%
    \def\ss{\xtextsansroman}
%
% Setting up font families
% % family 0
    \textfont0=\xmathtext
    \scriptfont0=\xxmathtext
    \scriptscriptfont0=\xxxmathtext
% family 1
    \textfont1=\xmathitalic
    \scriptfont1=\xxmathitalic
    \scriptscriptfont1=\xxxmathitalic
% family 2
    \textfont2=\xmathsymbol
    \scriptfont2=\xxmathsymbol
    \scriptscriptfont2=\xxxmathsymbol
% family 3   don't used in sup/sub scripts??
    \textfont3=\xmathext
    \scriptfont3=\xxmathext
    \scriptscriptfont3=\xxxmathext
% family 4
    \textfont\itfam=\xtextitalic
    \scriptfont\itfam=\xxtextitalic
    \scriptscriptfont\itfam=\xxxtextitalic
% family 5
%     \newfam\slfam \def\sl{\fam\slfam\tensl} % \sl is family 5
    \def\sl{\fam\slfam\tensl} % \sl is family 5
% family 6 (plain codes)
    \textfont\bffam=\xtextbold
    \scriptfont\bffam=\xxtextbold
    \scriptscriptfont\bffam=\xxxtextbold
% family 7
    \textfont\ttfam=\xtextteletype
    \scriptfont\ttfam=\xxtextteletype
    \scriptscriptfont\ttfam=\xxxtextteletype
% Font switches
    \def\rm{\fam0\xtextroman}%
    \def\mit{\fam\@ne}%
    \def\oldstyle{\fam\@ne\xmathitalic}%
    \def\cal{\fam\tw@}
    \def\it{\fam\itfam\xtextitalic}%
    \def\bf{\fam\bffam\xtextbold}%
%     \def\bfit{\fam\bolditalicfam\textbolditalic}%
    \def\bfit{\textbolditalic}%
    \def\tt{\fam\ttfam\xtextteletype}%
% Baselineskip, parindent, etc
    \normalbaselineskip=#7\p@% approx 3pt leading
    \smallskipamount=0.25\normalbaselineskip plus 1.5\p@ minus 1\p@%
    \medskipamount=0.5\normalbaselineskip plus 3.375\p@% minus 2\p@%
    \bigskipamount=#7\p@ plus 3\p@ minus 2\p@%
% Math 
    \abovedisplayskip=0.5\normalbaselineskip plus 0.25\normalbaselineskip
    \abovedisplayshortskip=0pt plus 0.25\normalbaselineskip
    \belowdisplayskip=0.5\normalbaselineskip plus 0.25\normalbaselineskip
    \belowdisplayshortskip=0.5\normalbaselineskip plus 0.25\normalbaselineskip
% Rest
    \parindent=\normalparindent % should no change
    \parskip=0\p@ plus 0\p@%
    \normallineskip=1\p@%
    \normallineskiplimit=\z@%
    \splittopskip=9\p@% strutt height should not change too??
    \setbox\strutbox=\hbox{\vrule height9\p@ depth5\p@ width\z@}%
    \normalbaselines\rm}

\def\roman#{\bgroup\rm\let\next= }
\def\bold#{\bgroup\bf\let\next= }
\def\ttype#{\bgroup\tt\let\next= }
\def\italic#1%
    {\begingroup
     \def \next
         {\def \next
              {\setbox 0 = \hbox {\nexttoken}%
               \ifdim \ht 0< \fontdimen 5 \font
                  \it #1%
               \else
                  \it #1\/%
               \fi}% setbox
          \ifcat A\noexpand \nexttoken
             \next
          \else 
             \ifcat 0\noexpand \nexttoken
                \next
             \else
                \ifcat \space \noexpand \nexttoken
                \else
                %%% \message {?: \meaning \nexttoken}%
                \fi
                \it #1\/%
             \fi
          \fi
          \endgroup}% next
     \futurelet \nexttoken \next}

\def\bolditalic#1%
    {\begingroup
     \def \next
         {\def \next
              {\setbox 0 = \hbox {\nexttoken}%
               \ifdim \ht 0< \fontdimen 5 \font
                  \it #1%
               \else
                  \bfit #1\/%
               \fi}% setbox
          \ifcat A\noexpand \nexttoken
             \next
          \else 
             \ifcat 0\noexpand \nexttoken
                \next
             \else
                \ifcat \space \noexpand \nexttoken
                \else
                %%% \message {?: \meaning \nexttoken}%
                \fi
                \bfit #1\/%
             \fi
          \fi
          \endgroup}% next
     \futurelet \nexttoken \next}

\catcode`\@=12

% GUSTLIB: verbatim.tex
%
% 1. \verbatim<character><text without that character><character>
%    causes the text to be set verbatim using \tt font; 
% 2. if the text uses up all the alphabet, \doubleverbatim macro 
%    can be used instead; this is similar to the previous one but 
%    a pair of characters is now used as a delimiter; 
% 3. in case of emergency \tripleverbatim macro may be of help... 
%

\def\uncatcodespecials % see D.E.K., pp. 344 and 380
    {\def\do##1{\catcode`##1=12}\dospecials}%
{\catcode`\^^I=\active \gdef^^I{\ \ \ \ }% TAB character is replaced by
                                         % 4 spaces; it is better than
                                         % nothing, but it does not mimic
                                         % true tabbing satisfactorily---maybe
                                         % some nice day...
 \catcode`\`=\active\gdef`{\relax\lq}}% this line inhibits Spanish 
                                      % ligatures ?` and !` of \tt font
\def\setupverbatim % see D.E.K., p. 381
    {\tt %
     \spaceskip=0pt \xspaceskip=0pt % just in case...
     \catcode`\^^I=\active %
     \catcode`\`=\active %
     \def\par{\leavevmode\endgraf}% this causes that empty lines aren't 
                                  % skipped
     \obeylines \uncatcodespecials \obeyspaces}%
{\obeyspaces \global\let =\ }% this causes that leading blanks aren't 
                             % skipped; cf. also def's of \space, \endgraf,
                             % \lq, \obeyspaces, and \obeylines, 
                             % D.E.K., pp. 351--352
% see D.E.K., p. 382
\def\doverbatim#1{\def\next##1#1{##1\endgroup}\next}%
\def\verbatim{\begingroup\setupverbatim\doverbatim}%
\def\dodoubleverbatim#1#2{\def\next##1#1#2{##1\endgroup}\next}%
\def\doubleverbatim{\begingroup\setupverbatim\dodoubleverbatim}%
\def\dotripleverbatim#1#2#3{\def\next##1#1#2#3{##1\endgroup}\next}%
\def\tripleverbatim{\begingroup\setupverbatim\dotripleverbatim}%

\catcode`\|=13
\def|{\begingroup\setupverbatim\doverbatim|}%

\endinput

% #1, #2, #3 -- tekst
% #4, #5, #6 -- math
% #7         -- interlinia
\fontsetup{12}{10}{8}{12}{10}{8}{15}
